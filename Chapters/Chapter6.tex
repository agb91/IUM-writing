% Chapter 3

\chapter{Functional Analysis} % Main chapter title

\label{Chapter3} % For referencing the chapter elsewhere, use \ref{Chapter1} 

%----------------------------------------------------------------------------------------

\section{Scenario-based requirements analysis}

Following there are a list of scenarios in which a user achieves a goal by doing a list of steps. The goal of these scenarios is to show in detail how the interaction between the user and the system takes place.

\begin{enumerate}

\item The user wants to analyse the runs between 30000 and 30010, plus the run 31456, to make a confrontation, he is interested both in the text-output and in the images: 
the user goes to the homepage, he is redirected to the authentication page and he do the login. If successful he returns automatically in the homepage, and he can insert the runs between 30000 and 30010 manually separating them with a semicolon or better clicking the "add range of runs" button, that opens a modal, in which the user can insert the minimum and the maximum of the range, and confirm (confirmation redirects the user to the homepage). After the user can add manually the run 31456 separating it from the others with a semicolon. If the inserted run numbers make doesn't make sense the system show on the page an error message. If the inserted run numbers are valid, the user can click the "send" button (before the button was red and un-clickable, now it is green and clickable) and start gAn. A progress bar is shown, a waiting message appears, and the user waits for some seconds (at this stage of development it is very hard to predict how much time gAn requests to execute). After some seconds the user is redirected in a page that shows the textual results: on the top of the page there is a navbar that shows the computed run numbers: the user uses this navbar to chose what part of the results to show in the screen. This bar is draggable, the user can freely move it. From this page the user can, through the button "show all images", access to another page dedicated to the images. In this page he can configure through dropdown menus the dimension, the layout, the group of the images (each image belongs to a group, the group depends on the sensor that generates the data from that the image is generated) to show. He can also, by clicking on a single image, access to another page, with a single image (the clicked image) that is completely accessible: the user can rotate the image, move it in a 3d space, zoom in, zoom out, select part, do some basic digital image processing and chose the kind of chart to show.   

\item The user wants to use the version 5.34 of Root (an old but stable version) to execute gAn: 
he complete the login, in the home page there is a button name "Chose Root version", clicking on this button the user is sent to a page where, he is informed about the current version of Root, and through a dropdown menu can chose among some version of Root already installed on the server (all the acceptable Root versions are installed on the server) . If the 5.34 version is one of the installed version he can select it and confirm, the goal is achieved. If the 5.34 version is not installed, the user cannot use this version.  

\item The user wants to select the "Rug-dev" branch of gAn:
the process is very similar to the process that allows the user to chose a Root version. The user complete the login, in the home page there is a button name "Chose gAn version", clicking on this button the user is sent to a page where, he, through a dropdown menu, can chose among some branches of gAN existing on the machine. "Rug-dev" is one of these, the user can confirm and the task is completed.

\item The user wants to make the computation using only the data taken from the sensor named "Mimito":
The user, after the login, in the homepage can use the button "Edit Config" the reach a page in which, through some dropdown menus, he can change the configuration file of gAn. Each of the dropdown menus is related to a sensor (often they are 5-6, it depends on the branch), and the option of the dropdown are "yes" or "no": if "yes" is selected the sensor's data are used in the computation, if "no" they aren't. One of the dropdown menus is named "Mimito", the user select "yes" for this sensor, "no" for all the others.  

\item %TODO %TODO %TODO %TODO %TODO %TODO %TODO %TODO %TODO %TODO %TODO %TODO %TODO %TODO %TODO %TODO INSERT SOMETHING RELATED TO THE ROOT LIKE IMAGE

\item The user want to download all the images related to the runs 40001 and 40002:
He can, after the login, insert the runs in the homepage (it is possible both by input field and by range selector), run gAn, wait the end of the execution, click "Show all images", and from here click "download all images". All the images will be downloaded in png format. %TODO %TODO %TODO %TODO %TODO %TODO %TODO  CHECK IF THIS PROCESS IS EXACTLY similar to what is written

\item The user wants to download the semi-processed root file related to the runs 31111 and 31112:
The steps are the same as the steps used to download the images, but instead of the button "Show all images", the user has to use the button "download root files".     

\end{enumerate}  

