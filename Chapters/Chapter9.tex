% Chapter 9

\chapter{Tests with the Users} % Main chapter title

\label{Chapter9} % For referencing the chapter elsewhere, use \ref{Chapter9} 

In this chapter first of all is exposed how the users are defined and what are their characteristics. Also, here are described the test conducted with them and in which ways they are useful to improve the gAn Web's features.   

\section{User's Profile: why?}

Generically a correct understanding of the user is very important to adapt the features of an application to him; In this case is also more important, because the kind of users to that this application is intended to is quite particular.


\section{User's Profile: classification}

\subsection{Generic Characteristics}
In this part of the document we can try to classify the user:
The estimated number of users of the application is around 15.
All the users of gAn Web works as "Shifters" in the AEgIS experiment: so they work with data acquisition and (simple) data analysis.
Usually the user is quite young: around 30-35 years. This is because the most of the time personnel with a major seniority doesn't work with tasks related to data acquisition. 
Half of them are male, and half female: in this particular case we think that it is not relevant in their interaction with gAn Web.
A considerable amount of the users use eyeglasses, and probably some of them pass too many time in front of a screen, but the kind of interaction designed with gAn is usually fast and isn't a big commitment for human eyes. We tried to avoid too many white backgrounds and the use of dazzling colours, but for the images' background they are considered acceptable to improve contrast and let the image to be more clear.  

\subsection{Education Background}
The cultural and educational level of the user is very high: almost all of them have a Phd, and they do a complex scientific job.

All of them have an extraordinary deep understanding of the particles physics: in particular they understand the technical terminology, the jargon in which the logbooks are written (because all the personnel, in shifts, writes the logbooks) and the names of the sensors (this is quite important, each sensor has a proper name, they are very numerous, and the fact that the users can identify them easily is a big ad advantage). 

Other important information: almost every time a user use gAn he is quite sure about what he is looking for. Situation in which the user searches a run number and reads generically the output of an analysis (or thinks about which is the correct analysis to use) is not common: in the most of the use cases the user is pretty sure about what is the information to search and what is the correct analysis. 

A different approach exists for the run number, in this case we must identify two different situation:

\begin{enumerate}

\item The most of the times (around 2/3) the user works with the last run, or the one immediately before. In this case is quite simple identify the correct run.
\item Sometimes the user searches information regarding a past run, that he remember to be interesting for some reason. In this case often the user remembers the date (the precise day, or the week) of the interesting run, but not the run number exactly.

\end{enumerate}

Most of the users are used to use the Linux's terminal, so they are used to quite complex but efficient interaction. A good advantage is that the users are very used to approach new applications, even complex ones, and they learn very rapidly: the advantage is that they are used to give developers hints about what they expect from an new application, because they do it quite often. Actually, the users, in particular the pilot users, are almost co-developers. GAn Web offers them a way to interact efficient like the terminal but easier (hopefully). They also are used to have a feedback for every action they do (using the Linux's terminal you can always know if the program is running, if it is crashed, and what are its outputs).
 
All the users have an excellent proficiency in English (they are used to speak English in their work environment). 

\subsection{User's psychological characteristics }
The style of reasoning is quite deductive: the users seem to act in a very scientific way: they observe the situation, they make hypothesis and they experiment them to check if they are true. In gAn Web we can observe that when they are for the first time in front of the application they observe the screen, reading all the tooltips, they understand (of better, they make hypothesis about) what each function do, and they test the functionalities making comments regarding if the application's behavior is the expected one. According the users  propensity the designer of gAn Web tries to let the application acts like the user expects.   

%TODO %TODO%TODO
%TODO
%TODO
%TODO

 caratteristiche
psicologiche, esperienza

Stile cognitivo Astratto/Concreto
Deduttivo/Induttivo/Abduttivo
Verbale/Spaziale
Analitico/Intuitivo
Attitudine (dell’utente rispetto al sistema) Positiva
Neutrale
Negativa
Motivazione (dell’utente rispetto all’uso) Alta
Moderata
Bassa

L’attivazione emotiva
Legge di Yerkes-Dodson 


Livelli semi-analfabeta
di alfabetizzazione livello medio
(lettura) buon livello
Abilità bassa
dattilografica media
(articolatoria) alta
Titolo media
di studio superiore
laurea
Linguaggio inglese
nativo altro
Alfabetizzazione bassa
Informatica moderata
alta


Si parla in genere di utente “novizio” e utente
“esperto”
Si possono distinguere anche in questo modo:
Novizio (novice): conosce poco del compito e
dell’applicazione
“First time”: conosce il compito ma non l’applicazione
Knowledgeable intermittent user: conosce il compito, ha
ampia conoscenza delle applicazioni software, ma ha difficoltà
a ricordare menu, funzionalità, etc.
Expert frequent user: conosce compito e applicazione
Esistono anche i “super-users” (o “power users”):
spesso svolgono un’importante funzione di
collegamento fra utenti “regolari” e i nuovi sistemi
informatici introdotti (e i loro sviluppatori) 

Caratteristiche ambientali 

\section{How the tests take place} 

\section{Results}
