% Chapter 1

\chapter{Early version} % Main chapter title

\label{Chapter3} % For referencing the chapter elsewhere, use \ref{Chapter1} 


At this point is important to analyse which are the requirements of gAn web at the first version:

\section{Functional requirements}
The definition of functional requirements aims to specifying in detail what the web application can do, and in which way an user can use it. The development of gAn web is divided in three stages, so the requirements are peculiar and different for each stage. The first version (from now "gAn web v1") is a very simple application: instead of access the program by a linux terminal like gAn, in gAn web v1, the user can use the program through a graphical interface. 
The requirements of this version are the following:

\begin{enumerate}

% 1
\item The user, in the homepage, can chose the run (only one run, for the moment) in which he is interested, using a input field. This field has validator, able to understand if the run number is inserted, if it is effectively a number, and if it is in an acceptable range. The user receives an explaining and precise error message directly on the homepage if the input field is empty and if the inserted value is not acceptable. It is always possible validate the inserted run and ensure that the related Root file exists? No, because in some moments some sensors don't work properly and it is inevitable that some root files related to some runs are incomplete, or even inexistent.

% 2
\item At this moment there is only a generic idea about what kind of analysis can be useful for the user, so in this version there will be only a generic analysis, able to dump in text all the possible information, and a group of example images (this is useless for the goal of the scientific research, but very useful to improve the understand the needed features of the web interface).
At this stage is not clear if the type of analysis will be chosen by the user of automatically selected by the program , so in gAn web v1 there are no buttons able to allow the user to select the type of analysis (this point will be reconsidered in next versions).

% 3
\item The user can start the program with a single click, by a button (usable only if the inserted number is valid).

% 4
\item When the program is executed the user can see the text output on the screen. This text is clear for a physicist (it is not clear for a person who doesn't have a specific preparation). 

% 5
\item When the program is executed the user can see the output images by clicking a button that link to a images-page. The images are ordered and organized by groups (the groups are related about which sensor takes the information necessary to create the image). The user can decide if he prefers to see the image in a little, medium or big format. The user can also decide if the images are distributed in the screen vertically or through a "carousel layout". The user can access the image in full-screen by clicking on it: he is redirected to a page with the image shown in full screen, and can return back to the all-images page by a return button. It is absolutely important to understand exactly which information are interesting for the users, and show in the images only them and all of them, this point will be solved with a confrontation with the pilot-users. 

%6
\item The user can modify a configuration file (a .txt file on the server), by a web interface. In this files there are some values the need to be setted (otherwise it uses default values), and the user can do it by radio buttons (in this way he is forced to chose valid values). This configuration file can modify the way in which gAn works and modify the resulting output (both the text and the images).   

\end{enumerate}

 
\section{Non-functional requirements}

All the versions have some non-functional requisites:

\begin{enumerate}

\item The first is quite simple: gAn web has to ensure that in case of crash of the program the web server mustn't crash too. The point is that on this web server (Apache server, installed on Linux) there are some other important applications, so, if gAn web crashes it is not a big problem, but the crash cannot force Apache, or worst the entire machine, to stop or restart. 
This requisite is quite easy to meet: a modern web application based on Html, Javascript, PHP and CSS is quite safe, a general crash of the server it is very unlikely to happen. If the C++ application or some Root libraries crashes (for example if the user asks for an inexistent run) the web application gracefully warn the user about the problem, but without uncontrolled behaviours.  

\item The application must work without install nothing. Also this requirement is very easy to meet: gAn Web is a web interface, it requires only a browser, nothing else.

\item The application must be compatible with any machine (except mobile phones, not requested), regardless of hardware, operating system, installed software. Also this requirement is achieved because of gAn Web only needs a browser to be used. 

\item The application must be easy to be modified and extended in the future by persons who aren't necessarily software engineers. The point is that the student who wrote this program is a "momentary collaborator" in the AEgIS experiment, and all the modifications to the program must be done by other people, in most cases physicists. So the best way is to comment in detail the code and keep the code simple (this is a basic good-programming requirement).   


\end{enumerate}

\section{Scenario based functional analysis}
TODO

\section{Prototypation}
TODO 