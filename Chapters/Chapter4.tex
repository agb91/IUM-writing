% Chapter 1

\chapter{Intermediate version} % Main chapter title

\label{Chapter4} % For referencing the chapter elsewhere, use \ref{Chapter1} 

In this chapter the features of the intermediate version are exposed.

\section{Functional requirements}

The intermediate version (from now "gAn Web v2") is more complex than the first one. It was born from the tips and the observation of the pilot users. The modifications are not numerous, but there are a lot of additions of new features. All the new required characteristics are exposed following:

\begin{enumerate}

% 1
\item The user can insert multiple runs: separated by a semicolon (but in case of errors the system can automatically correct them replacing symbols like "-" or "," or "." with semicolons and giving a more robust service). These runs can be inserted by an input field of by a range select button: this button open a "modal" that allows the user to chose the first run and the last, and automatically insert the comprised runs (for example, if the user inserts 30000 and 30010 the system inserts automatically all the run numbers between 30000 and 30010). This modal has a validation system, that ensure the correctness of the inserted values. It is not perfectly clear if this solution fit the needs of the users, but the tests with the whole users group will probably solve this doubt. 

% 2
\item The user can chose which kind of analysis execute. At this point the different analysis are related to the different branches of gAn that at this stage an heterogeneous group of programmers are developing and uploading on Gitlab. It is no clear which of these branches will be definitive and which no, lo the program must be able to use all of them. The type of the analysis depends on the version of gAn downloaded and used for the execution of the program. In gAn Web v2 five complete branches exist, but in the future they can became more. They are externally very similar, the differences are the algorithms in the program, but they give a different output (different output but in the same format: text and images). 
At this stage is not clear if all this different versions will be used for the final application, to clarify this point the best solution is observe directly the user's behavior. 

% 3
\item The configuration file is not only in text format, but also can be in xml format (it depends on the selected version of gAn). The xml ensures a stronger structure, and must be transparent to the user (he mustn't see differences between the configurator that works with a txt file and the one that works with xml). At this stage both text format and xml format are acceptable, to ensure the retro-compatibility of some analysis, but is possible that in further versions the xml-based design will became dominant. 

%4
\item The user can chose what version of Root he wants to use for the program. Theoretically different versions of Root are perfectly compatible, but in practice each version of gAn is designed to work with a particular version of Root and to avoid problems it seems to be a good idea to allows the user to choose freely which version of Root use among the installed versions on the server.  

%5
\item The user can save images on his hard disk: he can chose from the shown images in the images page an image to download by clicking on a specific download button near the image. Furthermore there is another button "Download All" with whom the user can simple download all the output images.

%6 
\item The user can download a reduced version of the root file with informations about the images and the results: gAn produce this kind of files as "half-processed" during the computing, and it is not a problem to save this on the hard disk of the server in a specified folder. For an expert user can be scientifically interesting have this file (this root file contain more information than the output, the most of this information is useless [it is an "half-processed" file] , but sometimes an expert user can find something interesting), so the user must have the opportunity to download this.     

% 7
\item The first little group of user prefers the dropdown menu to the radio button, so all the radio buttons in the program are replaced by dropdown menus.

% 8
\item The user has to access not only to a png image, but to a root-image. This kind of image is interactive: the user can with a left click of the mouse (a continued click, like the "dragging") select parts of the image and zoom them, and with a right click do dynamically some kind of image processing (set colors, chose  what kind of chart to show, modify the chart legend, translate in a 3D space the image etcetera). In this way each user can choose freely which information he wants see in the image (try to overseen user's needs in this part seems to be too difficult and not useful).
All of this must be done by the user through a browser window. This requisite seems to be quite complex, but Root provides libraries (these libraries work well but they are poorly documented) to interact with Javascript, and can in some way resolve the problem.  

% 9
\item In the homepage the user can see the run number of the last root file produced by the machine, and its creation date and time (so, he can understand what is the maximum of the range of the insertable numbers). Also, the run number is an unit of measurement of the time, so through this number the user can have information about the progress of the experiment.  

In the intermediate version there was another functional requisite: ideally the user should have been able to select a gAn version also if not installed in the server machine: in this case the system should have been capable to automatically search on the AEgIS Gitlab repository the correct version (if existing), download it, unpack it in the server, and use it to execute the program. 
After some discussion this requirement has been cancelled, because it was considered complex, basically useless, and potentially harmful (on the branches of the repository there are untested and incomplete versions, that can create if executed wrong outputs, so wrong scientific results). At this moment installing manually the stable versions of gAn on the server seems to be a more smart way to work.

% 10
\item There is a login system: the user must insert the password of the office to use the system. The authentication is based on the confrontation between the hash function of the inserted password and the hash function of the AEgIS password. If the password is correct the user receives a cookie, before each action in the site the server request and check this cookie to be sure about the identity of the user.  


\subsection{Ambiguities (and related solutions)}

At least a point seems to be quite ambiguous: 

The textual output of gAn needs to be formatted in some way to be more organized and clear? 
The answer is difficult: for a non-physicists this output seems to be disordered, too long, with too many groups of informations, and very difficult to understand, but on this question the pilot users (that are physicists) questioned answered that the output is perfectly clear and doesn't need to be modified or improved in any way. The only requests of the users were about the font and the font-size. To check this fact the best solution probably is observe the behavior of the users at work, and eventually ask them informations about that.
Anyway, in the second version, in case of multiple run selection, there is a "navbar" that allows the user to show only a run-result per time.

\end{enumerate}


\section{Scenario based functional analysis}
TODO

\section{Prototypation}
TODO 
